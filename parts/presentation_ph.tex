% Définition du Process Hitting + sortes coopératives

\begin{frame}
  \frametitle{The Automata Network / Process Hitting modeling}
  \framesubtitle{\tcite{Paulev\'e et al. 2012}}

% 1 : Sortes
\only<1>{
\tikzstyle{process}=[circle,minimum size=15pt,font=\footnotesize,inner sep=1pt]
\tikzstyle{tick label}=[color=white, font=\footnotesize]
\tikzstyle{tick}=[transparent]
\tikzstyle{hit}=[transparent]
\tikzstyle{selfhit}=[transparent, min distance=30pt,curve to]
\tikzstyle{bounce}=[transparent]
\tikzstyle{hlhit}=[transparent]
\begin{center}\scalebox{\scaleex}{
\begin{tikzpicture}
\exphdef
\end{tikzpicture}
}\end{center}
}

% 2 : Processus
\only<2>{
\tikzstyle{process}=[circle,draw,minimum size=15pt,font=\footnotesize,inner sep=1pt]
\tikzstyle{tick label}=[font=\footnotesize]
\tikzstyle{tick}=[densely dotted]
\tikzstyle{hit}=[transparent]
\tikzstyle{selfhit}=[transparent, min distance=30pt,curve to]
\tikzstyle{bounce}=[transparent]
\tikzstyle{hlhit}=[transparent]
\begin{center}\scalebox{\scaleex}{
\begin{tikzpicture}
\exphdef
\end{tikzpicture}
}\end{center}
}

% 3 : États
\only<3>{
\tikzstyle{hit}=[transparent]
\tikzstyle{selfhit}=[transparent, min distance=30pt,curve to]
\tikzstyle{bounce}=[transparent]
\tikzstyle{hlhit}=[transparent]
\begin{center}\scalebox{\scaleex}{
\begin{tikzpicture}
\exphdef

\TState{3}{a_0,b_1,z_0}
\end{tikzpicture}
}\end{center}
}

% 4 : Actions
\only<4->{
\tikzstyle{tick}=[densely dotted]
\tikzstyle{hit}=[->,>=angle 45]
\tikzstyle{selfhit}=[min distance=30pt,curve to]
\tikzstyle{bounce}=[densely dotted,>=stealth',->]
\tikzstyle{hlhit}=[very thick]
\begin{center}\scalebox{\scaleex}{
\begin{tikzpicture}
\exphdef
\TState{4}{a_0,b_1,z_0}
\TState{5}{a_0,b_1,z_1}
\TState{6}{a_1,b_1,z_1}
\TState{7}{a_1,b_1,z_2}
\end{tikzpicture}
}\end{center}
}

%\medskip
\begin{liste}
  \item \tval{Automata/Sorts}: components \qex{$a$, $b$, $z$}
\pause[2]
  \item \tval{Local states/Processes}: levels of expression \qex{$z_0$, $z_1$, $z_2$}
\pause[3]
  \item \tval{States}: sets of active processes%
  \only<3-4>{\qex{$\PHetat{a_0, b_1, z_0}$}}%
  \only<5>{\qex{$\PHetat{a_0, b_1, z_1}$}}%
  \only<6>{\qex{$\PHetat{a_1, b_1, z_1}$}}%
  \only<7>{\qex{$\PHetat{a_1, b_1, z_2}$}}%
\pause[4]
  \item \tval{Transitions/Actions}: dynamics \qex{\only<4>{\underline}{$\PHfrappe{b_1}{z_0}{z_1}$}, \only<4-5>{\underline}{$\PHfrappe{a_0}{a_0}{a_1}$}, \only<6>{\underline}{$\PHfrappe{a_1}{z_1}{z_2}$}}
 \pause[5]
   \item \tval{Stochastic parameters}: Hybrid Modeling
\end{liste}
\end{frame}


\begin{comment}
\begin{frame}
  \frametitle{The Process Hitting modeling}
  \framesubtitle{\tcite{PMR12-MSCS}}

\begin{center}\scalebox{\scaleex}{
\begin{tikzpicture}
\exphcoop
\end{tikzpicture}
}\end{center}

\only<-14>{
\begin{liste}
  \item How to introduce some \tval{cooperation} between sorts? \qex{$\PHfrappe{a_1 \wedge b_0}{z_1}{z_2}$}
\pause[4]
  \item Solution: a \tval{cooperative sort} \qex{$ab$} \only<12->{\quad to express \qex{$a_1 \wedge b_0$}}
\pause[8]
  \item Constraint: each configuration is represented by one process \qex{$\PHetat{a_1,b_0} \pause[11]\Rightarrow ab_{10}$}
\pause[14]
  \item Advantage: regular sort; drawbacks: complexity, temporal shift
\end{liste}
}

\only<15->{
The Process Hitting framework:

\smallskip
\begin{itemize}
  \item \tval{Dynamic} modeling with an \tval{atomistic} point of view
  \item Efficient \tval{static analysis} (fixed points, reachability)
  \item Possible extensions (stochasticity, priorities)
  \item Useful for the study of \tval{large biological models}
\end{itemize}
}
\end{frame}
\end{comment}
