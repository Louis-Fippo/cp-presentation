\begin{frame}
\frametitle{Definition of bifurcation}
\begin{figure}[t]
\centering
\begin{tikzpicture}[node distance=1.5cm]
\node (s0) [circle,fill=gray!60] {$s_0$};
\node (sb) [circle,fill=gray!60,above right of=s0,xshift=2cm] {$s_b$};
\node (g) [circle,fill=blue!20,minimum width=7mm,below right of=sb,xshift=2cm] {$S_{g_1}$};
\node (su) [above right of=sb] {$s_u$};
\node (u) [below right of=su,xshift=5mm] {$\mathbf{\color{red}x}$};
\path[->] (sb) edge[->,red,thick] node[auto] {$t_b$} (su) ;
\path[->,bend left=20,densely dashed]
	(s0) edge (sb)
	(sb) edge (g)
	(s0) edge [bend right=10] (g)
	(su) edge[>={Rays[length=3mm,width=3mm]}] (u)
;
\end{tikzpicture}
\end{figure}

\begin{block}{Definition}
\centering
$t_b$ is a bifurcation transition from $s_0$ to $g_1$

\smallskip
$\Longleftrightarrow$
there exists $s_b$ such that:
%rajouter g1 \in Sg
\begin{align*}
\text{\cI}\enspace &s_u\nreach g_1
&
\text{\cII}\enspace &s_b\reach g_1
&
\text{\cIII}\enspace &s_0\reach s_b
\end{align*}
\end{block}
\end{frame}
